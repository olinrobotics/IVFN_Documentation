\section{The Current Robotic Tuna}

The current robotic tuna is shown in the figure below: \\ \\

\begin{figure}[h!]
\centering
\includegraphics[scale=.08]{wholefish.jpg}
\caption{The current version of the robotic tuna}
\label{fig:wholefish}
\end{figure} 

\noindent For the research this semester, we were particularly interested in the efficiency of the propulsion system for the fish. The propulsion system designed in the summer of 2014 is shown in the figures below:

\begin{figure}[h!]
\centering
\begin{subfigure}{.5\textwidth}
  \centering
  \includegraphics[scale=.04]{wholetail.jpg}
  \caption{The whole tail section}
  \label{fig:wholetail}
\end{subfigure}%
\begin{subfigure}{.5\textwidth}
  \centering
  \includegraphics[scale=.17]{twoservos.jpg}
  \caption{Close-up view of the two servos powering the tail section}
  \label{fig: twoservos}
\end{subfigure}
\caption{Views of the tail section of the current version of the robotic tuna}
\label{fig:tailviews}
\end{figure}

\begin{figure}[h!]
\centering
\begin{subfigure}{.5\textwidth}
  \centering
  \includegraphics[scale=.05]{topservo.jpg}
  \caption{Propulsion servo}
  \label{fig:topservo}
\end{subfigure}%
\begin{subfigure}{.5\textwidth}
  \centering
  \includegraphics[scale=.05]{bottomservo.jpg}
  \caption{Steering Servo}
  \label{fig: bottomservo}
\end{subfigure}
\caption{Close-up views of the servo that delivers propulsion power (left) and the servo that aids in steering (right)}
\label{fig:servoviews}
\end{figure}

\noindent In essence, the tail designed in the summer of 2014 had two servos powering it. One servo was mounted to the base of the tail and this servo served as the main source of power for propelling the fish forward. This servo essentially performed the function of a propeller on a typical boat. The other servo was mounted in the fourth section of the tail and was used to control only the last three sections of the tail to help steer the fish where necessary. This servo essentially performed the function of a ruder on a typical boat. \\ \\
%
Although this design was useful as a proof-of-concept, using the servos in this manner is extremely inefficient. As such, in order to advance toward the goal of a highly efficient robotic tuna, we wanted to design a significantly more efficient tail.